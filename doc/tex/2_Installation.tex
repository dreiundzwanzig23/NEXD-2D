\chapter{Requirements and Installation}
\label{chap:installation} % Always give a unique label
% use \chaptermark{}
% to alter or adjust the chapter heading in the running head
To install and run the software, a number of requirements have to be met.
\iffalse
\section{System requirements}
\label{sec:seytemreq}
    Currently the software package has been tested on MacOS X Yosemite and above and on a Debian Linux System \todo{specify kind and version}. Currently installation on Windows has not been tested and is subsequently not supported.
    As the software is designed to be run on multi-cpu environments minimum system requirements are hard to define. If you want to run the examples on your local machine the following specifications are recommended\footnote{These specs are based on the minimum we developers used during testing. It may work with different (lower) specs but it is untested.}:
    \begin{itemize}
        \item CPU: 2,6 GHz Dual-Core Intel Core i5 Processor (With hyperthreading allows for simulations on four cores) or equivalent
        \item RAM: 8 GB 1600 MHz DDR3L
        \item Harddrive: At least 1GB free space
    \end{itemize}
    \textbf{Attention:} Be aware that simulation speed is substantially influenced by the computer hardware. Lower system specifications may work but simulations take significantly longer. For larger simulations it is highly recommended that a multi-cpu computer system (e.g. small clusters) are used. 
    Single core calculations are no longer supported. 
\fi
\section{Software requirements}
\label{sec:softreq}
To install NEXD the user needs the following software installed on his system:
\begin{itemize}
    \item GNU Make (\url{https://www.gnu.org/software/make/}),
    \item Fortran compiler (GNU Fortran or Intel Fortran). The version 0.4 of this software has been tested using the following compiler versions:
		\begin{itemize}
			\item Intel Fortran version 19.0
			\item GNU Fortran version 4.9.2, 6.3, 7.4, 9.2 and 9.3
		\end{itemize}
	Version 0.2 of NEXD 2D has also been tested with:
		\begin{itemize}
			\item Intel Fortran version 17.0.4
			\item GNU Fortran version 5.4 and 7.3
		\end{itemize}	
    \item LAPACK libraries (\url{http://www.netlib.org/lapack/},
    \item MPI libraries, e.g. OpenMPI, 
    \item METIS (\url{http://glaros.dtc.umn.edu/gkhome/metis/metis/overview}, tested with version 4.0.3) for parallel applications,
    \item a Python installation.
\end{itemize}
 
A number of Python scripts are used during the installation of NEXD 2D and as an aid in the creation of input files. To run those, a Python version (recommended 2.7, or 3.6 or higher) is required. A good and easy way to get Python is to install it via the Anaconda distribution, which is available for Windows, MacOS and Linux. It can be downloaded under \url{https://www.anaconda.com/download/}.

\section{Installation}
To install the software, follow these steps:
\begin{enumerate}
    \item If not yet done, download the source code of the NEXD 2D main package by downloading it from \url{https://github.com/seismology-RUB/NEXD-2D}.
    \item Install all software dependencies.
    \item Adjust the software to your system and personal requirements by changing the Makefile appropriately (e.g., change the path to your METIS installation and set your compiler, see code listing \ref{lis:make2}).
    \item Compile METIS.
    \item Compile NEXD using the command ``make all'' in your console from your installation directory.
\end{enumerate}

\section{Changing the Makefile}
\label{subsec:changeMake}
    As mentioned in step 3 of the instruction on how to install the software, the Makefile needs to be changed at certain position. The following sections need to be adjusted by the user:
    
    \lstinputlisting[
        float=h, 
        captionpos=b, 
        label=lis:make1, 
        caption=Exemplary content of the Makefile.,
        firstnumber = 31, 
        firstline=31, 
        lastline=34,
        breaklines=true,
        postbreak=\mbox{\textcolor{red}{$\hookrightarrow$}\space}
    ]{source_files/Makefile.txt}
    Change these lines to select the compiler. The default compiler flags are set for optimum performance of the user and need not be adjusted. Recommended flags for developers are given by the comments in the Makefile. Please review the Makefile for more information.
    
    Please adjust the path to your METIS library as specified in the code listing \ref{lis:make2}.
    \lstinputlisting[
        float=ht, 
        captionpos=b, 
        label=lis:make2, 
        caption=Exemplary content of the Makefile.,
        firstnumber = 294, 
        firstline=294, 
        lastline=298,
        breaklines=true,
        postbreak=\mbox{\textcolor{red}{$\hookrightarrow$}\space}
    ]{source_files/Makefile.txt}
\newpage
\section{Clean installation directory}
\label{sec:cleansimdirc}
Below a clean tree for the installation directory with all sub-folders is shown.

{\footnotesize
\begin{forest}
 pic dir tree,
  where level=0{}{% folder icons by default; override using file for file icons
    directory,
  },
[NEXD2D installation directory
	[doc, label=right:{Contains all files related to the documentation, including the Latex source files}] 
    [simulations
        [2layer\_pml\_att
            [mesh, label=right:{Contains files related to the mesh}]
            [data, label=right:{Contains the files: parfile, source and stations}]
            [ref
               [att, label=right:{Contains reference seismograms with attenuation and without PMLs (42 files)}]
               [no\_att, label=right:{Contains reference seismograms without attenuation and without PMLs (42 files)}]
               [pml, label=right:{Contains reference seismograms without attenuation and with PMLs (42 files)}]
               [pml\_att, label=right:{Contains reference seismograms with attenuation and with PMLs (42 files)}]
            ]
        ]
        [example\_inversion
            [mesh, label=right:{Contains files related to the mesh}]
            [data, label=right:{Contains the files: parfile, source, stations and invpar}]
            [ref, label=right:{Contains reference seismograms and .vtk files (22 files)}]
            [inversion, label=right:{Contains the seismograms obtained from the real model}]
        ]        
        [example\_linearSlipInterface
            [mesh, label=right:{Contains files related to the mesh}]
            [data, label=right:{Contains the files: parfile, source, stations, interfaces and fracs}]
            [ref
                [noPML, label=right:{Contains reference seismograms with PML (60 files)}]
                [withPML, label=right:{Contains reference seismograms without PML (60 files)}]
            ]
        ]
        [example\_poro
            [mesh, label=right:{Contains files related to the mesh}]
            [data, label=right:{Contains the files: parfile, source and stations}]
            [ref, label=right:{Contains reference seismograms (178 Files)}]
            [tools, label=right:{Contains Matlab Code to calculate velocities and required element size.}]
        ]
        [lambs
            [mesh, label=right:{Contains files related to the mesh}]
            [data, label=right:{Contains the files: parfile, source and stations}]
            [ref, label=right:{Contains reference seismograms and a README (13 Files)}]    
        ]
    ]
    [src, label=right:Contains all fortran source files]
    [tools
        [scripts, label=right:Contains a number of python scripts]
    ]
    [LICENSE, file]
    [Makefile, file]
    [Readme.md, file] 
]
\end{forest}
}
\\
For the remaining part of the manual the NEXD 2D installation directory will be referred to as \url{dg2d}.
