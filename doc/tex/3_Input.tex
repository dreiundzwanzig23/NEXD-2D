\chapter{Input}
\label{chap:input}
There are three different parameter files for the basic version of the code. They contain all parameters necessary to run the program. These parameter files are stored in \url{yourSimulation/data/} and are called:
\begin{itemize}
	\item parfile - general parameters
	\item source - parameters regarding the sources
	\item stations - parameters regarding the receivers
\end{itemize}
Additional parameter files are necessary to apply the special features of the code. If fractures are to be included in the simulation, two additional files are required:
\begin{itemize}
	\item fracs - specifies the position of fractures
	\item interfaces - parameters regarding the fractures
\end{itemize}
If the code is used to perform a full waveform inversion, the following file is required:
\begin{itemize}
	\item invpar - parameters regarding the inversion
\end{itemize}
If poroelasticity is to be considered during the simulation, no additional parameter files are reqiured. However, the requirements for files described in Section \ref{sec:cubfiles} need to be fullfilled. The contents of the files described above will be explained in detail in subsequent sections.

\medskip
To run the simulation, a number of other files are required that are generated by the meshing program of the users choice. These files need to be stored in the directory \url{yourSimulation/mesh/} and are called
\begin{itemize}
	\item absorb
	\item coord
	\item free
	\item mat
	\item matprop
	\item mesh
\end{itemize}
These files have a specific structure that will be explained for each file in a separate section. For the supplied examples they have been created using a python script called \url{cubit2dg2d.py} which uses the output (\url{*.cub}) from the commercial meshing software Trelis (formerly CUBIT). In addition, gmsh-scripts (\url{*.geo}) are available for many examples, too. They can be used to create meshs (\url{*.msh}) and these can be converted by the python script \url{gmsh2dg2d.py}.

If a different meshing software should be used, the user can change the script \url{gmsh2dg2d.py} to create the input for NEXD. This might be straightforward since this script uses the library meshio, that provides readers for many mesh formats.

	\paragraph*{General Advice}
		Prior to discussing the contents of the parameter files some general advice is given to avoid mistakes (especially related to the input format). Parameters that appear as a certain type, e.g. integer, in the parameter files will be read in the same way. For integers that poses no issue. For floating point numbers (floats) there are a number of formatting options that will all be accepted. For example, 0.01 will be valid if entered in this way, but also if entered 1e-2. For boolean variables, either .false./.true. or F/T are valid. If any type error is detected an error message will be raised containing the affected parameter and the simulation will not run.
		
	\section{Files in \texttt{data}}
		The first section covers all parameter files that are located in \url{yourSimulation/data/}.
		
	\subsection{parfile}
	\label{subsec:parfile}
		This general parameter file contains all general parameters necessary to configure the programs. Here, all relevant features can be enabled/disabled. For convenience specific sections of the parameter file will be discussed separately.
		\subsubsection{General Parameters}
		\label{subsec:genpar}
			\lstinputlisting[
    float=ht, 
    captionpos=b, 
    label=lis:genpar,
    firstline = 1,
    lastline = 19,
    caption=General parameters,
    breaklines=true,
    postbreak=\mbox{\textcolor{red}{$\hookrightarrow$}\space}    
    ]{source_files/parfile.txt}
			The section of the parfile displayed in listing \ref{lis:genpar} shows a number of general parameters. 	
			\begin{itemize}
				\item \textbf{title}: Currently ``title'' is just used to provide a name for the simulation in the output log files.
				\item \textbf{fluxtype}: This parameter switches between two ways to calculate the numerical fluxes. Option ``0'' selects the elastic fluxes calculated by \cite{Moeller.2018} for the slip interface calculations. If these fluxes are selected and simultaneously attenuation is set to ``.true.'', a warning is raised. Option ``1'' selects the fluxes implemented by \citep{Lambrecht.2015}. These fluxes do not support a calculation including fractures.
				\item \textbf{log}: Set ``.true.'' if the log is to be created/displayed, ``.false.'' otherwise.
				\item \textbf{nproc}: Defines the number of threads (processors) used in the simulation. The value is specific to the users system configuration. Single core calculations are currently not supported, thus a minimum of 2 should be entered.
				\item \textbf{extvel}: Select if an external velocity model is to be used. This works currently only for an elastic model.
				\item \textbf{externalfilename}: Set the file name of the external velocity model. Must contain the location of the file in relation to \url{yourSimulation}, or the absolute file path. This parameter is ignored as long as ``extvel'' is set to ``.false.''. 
			\end{itemize}
		\subsubsection{Parameters regarding seismograms}
			This part of the parfile is displayed in listing \ref{lis:seispar}. There are no conflicts to other parameters.
			\lstinputlisting[
    float=ht, 
    captionpos=b, 
    label=lis:seispar,
    firstnumber = 20,
    firstline = 20,
    lastline = 26,
    caption=Parameters regarding seismograms.,
    breaklines=true,
    postbreak=\mbox{\textcolor{red}{$\hookrightarrow$}\space}    
    ]{source_files/parfile.txt} 
    		\begin{itemize}
    			\item \textbf{subsampling\_factor}: Can be used to reduce the sampling rate of the seismograms, where $\mathrm{sampling\_rate}=\mathrm{subsampling\_factor} \cdot \mathrm{dt}$.
    			\item \textbf{autoshift}: Corrects for the ``width'' of the used wavelet. For a Ricker or Gaussian the maximum will be at t=0. If set to .false. plott0 is used instead.
    			\item \textbf{autodt}: Offset for the seismogram.
    			\item \textbf{div}: ``.true.'' if the radial component of the seimogram is to be calculated, otherwise ``.false.''.
    			\item \textbf{curl}: ``.true.'' if the tangential component of the seimogram is to be calculated, otherwise ``.false.''.
    		\end{itemize}

    	\subsubsection{Movie parameters}
    	\label{subsubsec:moviepar}
			This part of the parfile is displayed in listing \ref{lis:moviepar}. These parameters influence the creation of binary files that are used by the program \url{movie} to generate ``vtk''-files (Visualisation Toolkit).
			\lstinputlisting[
    float=ht, 
    captionpos=b, 
    label=lis:moviepar,
    firstnumber = 27,
    firstline = 27,
    lastline = 38,
    caption= Movie parameters.,
    breaklines=true,
    postbreak=\mbox{\textcolor{red}{$\hookrightarrow$}\space}    
    ]{source_files/parfile.txt}
    		\begin{itemize}
    			\item \textbf{movie}: This parameter enables or disables the creation of binary files that can be used to create vtk-files.
    			\item \textbf{frame}: This parameter needs to be between 1 and the total number of time steps in the simulation. The parameter defines the number of time steps between individual snapshots for the output.
    			\item \textbf{save\_movie\_trimsh}: Create files with average values for each element. These files take significantly less space but also have a poorer resolution of the wave field.
    			\item \textbf{save\_movie\_points}: Create files with data for interpolation points within the mesh. \textit{Warning:} These files take significantly more space and slow down the simulation but also show a much higher resolution of the wave field.
    			\item \textbf{save\_movie\_displacement}: Enable to plot the displacement field.
    			\item \textbf{save\_movie\_velocity}: Enable to plot the particle velocity field.
    			\item \textbf{save\_movie\_stress}: Enable to plot the stress field.
    			\item \textbf{save\_movie\_p1}: Enable to plot the pressure field of the first fluid (only for poroelasticity).
    			\item \textbf{save\_movie\_v1}: Enable to plot the particle velocity field of the first fluid (only for poroelasticity).
    			\item \textbf{save\_movie\_p2}: Enable to plot the pressure field of the second fluid (only for poroelasticity).
    			\item \textbf{save\_movie\_v2}: Enable to plot the particle velocity field of the first fluid (only for poroelasticity).
    		\end{itemize}
		\subsubsection{Parameters regarding time integration}
			This part of the parfile is displayed in listing \ref{lis:timepar}. These parameters influence the selection of the method that is used for time integration, the number of time steps and the selection of the time step. 
			\lstinputlisting[
    float=ht, 
    captionpos=b, 
    label=lis:timepar,
    firstnumber = 40,
    firstline = 40,
    lastline = 48,
    caption= Parameters regarding time integration,
    breaklines=true,
    postbreak=\mbox{\textcolor{red}{$\hookrightarrow$}\space}    
    ]{source_files/parfile.txt} 
    		\begin{itemize}
    			\item \textbf{timeint}: This parameter selects the method for time integration. Options are:
  				\begin{enumerate}
  					\item Euler,
  					\item total variation diminishing (TVD) second order Runge-Kutta: rk2 (TVD),
  					\item third order TVD Runge-Kutta: rk3 (TVD),
  					\item low-storage five-stage fourth order Runge-Kutta, rk4 (LSERK).
  				\end{enumerate}
  				Enter the number to select the appropriate method.
  				\item \textbf{autont}: Can be used to automatically calculate the number of timesteps to reach the time specified in t\_total.
  				\item \textbf{nt}: Select the number of time steps for the simulation. Not used if autont is true.
  				\item \textbf{t\_total}: Select the total simulated time. Not used if autont is false.
  				\item \textbf{autodt}: Automatically calculate the time step based on the Courant-Friedrichs-Lewy (CFL) criterion.
  				\item \textbf{dt}: Specifies a dt different from the automatic one. This parameter is ignored as long as autodt is true. 
  				\item \textbf{cfl}: Parameter for the CFL stability criterion. Recommended values are: rk2, cfl = 0.4; rk4, cfl = 0.7.
  				\item \textbf{simt0}: Specifies the starting time of the simulation.
    		\end{itemize}

		\medskip
    		\fcolorbox{red}{white}{\parbox{\textwidth-2\fboxsep-2\fboxrule}{\textbf{Attention}: The recommended time integration method is the second order Runge-Kutta method (rk2) with a cfl of 0.4.}}

		\subsubsection{Parameters regarding PML}
			This part of the parfile is displayed in listing \ref{lis:pmlpar}. These parameters define the characteristics of Perfectly Matched Layers enhancing the absorbing boundary conditions.
			\lstinputlisting[
    float=ht, 
    captionpos=b, 
    label=lis:pmlpar,
    firstnumber = 50,
    firstline = 50,
    lastline = 59,
    caption= Parameters regarding PML,
    breaklines=true,
    postbreak=\mbox{\textcolor{red}{$\hookrightarrow$}\space}    
    ]{source_files/parfile.txt} 
    
    		\begin{itemize}
    	 		\item \textbf{set\_pml}: Enable PML additionally to absorbing boundary conditions. This parameter should only be set to ``.true.'' if the mesh has been designed with a PML layer (see section \ref{chap:model} for details).
    	 		\item \textbf{pml\_delta}: This parameter describes the thickness of the PML. A uniform thickness of the PML layer throughout the model is assumed. PML layers with non-uniform thickness are currently not supported.	 	
    		\end{itemize}
			The parameters below influence the dampening profile of the PML. It is advised to keep the suggested values.
			
			\begin{itemize}
				\item \textbf{pml\_rc}: Theoretical reflection coefficient for the PML. Default is 0.01.
				\item \textbf{pml\_kmax}: Maximum value for the variable $k$. Values may range between 1 and 20.
				\item \textbf{pml\_afac}: Factor for $\alpha_{max}$.
			\end{itemize}
			
			The following parameters are not directly related to the PML layer. They serve as a control mechanism to ensure that the PML are working properly. It is designed to detect instabilities inside the PML and disables the PML, if necessary. For most cases the parameters for the length of the trigger windows can be left to the default values.
			
			\begin{itemize}
				\item \textbf{use\_trigger}: Parameter to enable the STA-LTA trigger. 
				\item \textbf{avg\_window1}: Size of the window to determine the long term average (LTA).
				\item \textbf{avg\_window2}: Size of the window to determine the short term average (STA).
				\item \textbf{sta\_lta\_trigger}: Threshold for the STA-LTA trigger. Values above this threshold will lead to disabling the PML.
			\end{itemize}
			
		\subsubsection{Parameters regarding attenuation}
			This part of the parfile is displayed in listing \ref{lis:attpar}. If enabled, these parameters influence the calculation of the attenuation.
			\lstinputlisting[
    float=ht, 
    captionpos=b, 
    label=lis:attpar,
    firstnumber = 61,
    firstline = 61,
    lastline = 65,
    caption= Parameters regarding attenuation,
    breaklines=true,
    postbreak=\mbox{\textcolor{red}{$\hookrightarrow$}\space}    
    ]{source_files/parfile.txt}
    		\begin{itemize}
    			\item \textbf{attenuation}: Enables the simulation of viscoelastic attenuation. This requires specific values to be set in the \url{yourSimulation/mesh/matprop} file (see section \ref{subsec:matprop}).
    			\item \textbf{f0\_att}: Attenuation is simulated by a number of Maxwell bodies (see \cite{Lambrecht.2015} for details) to introduce disperion. This parameter sets the frequency to which the model parameters are assigned to.
				\item \textbf{f\_max\_att}: Defines the upper end of the frequency range where the quality factors are held approximately constant.
				\item \textbf{att\_factor}: Factor defining the lower end of the frequency range by f\_min\_att = f\_max\_att / att\_factor.
    		\end{itemize}
    		
		\subsubsection{Fracture parameters}
			This part of the parfile is displayed in listing \ref{lis:fracpar}. These parameters influence if and how the influence of fractures is created. For a detailed description on the theory behind these calculations, the interested user is referred to \cite{Moeller.2018}.
			\lstinputlisting[
    float=ht, 
    captionpos=b, 
    label=lis:fracpar,
    firstnumber = 67,
    firstline = 67,
    lastline = 70,
    caption=Parameters regarding fractures.,
    breaklines=true,
    postbreak=\mbox{\textcolor{red}{$\hookrightarrow$}\space}    
    ]{source_files/parfile.txt} 
    		\begin{itemize}
    			\item \textbf{lsi}: This parameter enables or disables the calculation of the influence of fractures. Enableling this feature requires the additional parameter files ``fracs'' and ``interfaces''. Currently this works not for calculation including attenuation.
    			\item \textbf{normal}: Enables the calculation of the fracture influence normal to the fracture plane.
    			\item \textbf{tangential}: Enables the calculation of the fracture influence parallel to the fracture plane.
    		\end{itemize}  
    		  		
    	\subsubsection{Parameters regarding poroelasticity}
			This part of the parfile is displayed in listing \ref{lis:poropar}. If this is enabeld, the materials are handled as poroelastic media. For a detailed description on the theory behind these calculations, the interested user is referred to \cite{Boxberg.2019}.
			\lstinputlisting[
    float=ht, 
    captionpos=b, 
    label=lis:poropar,
    firstnumber = 72,
    firstline = 72,
    lastline = 77,
    caption= Parameters regarding poroelasticity,
    breaklines=true,
    postbreak=\mbox{\textcolor{red}{$\hookrightarrow$}\space}    
    ]{source_files/parfile.txt}
    		\begin{itemize}
    			\item \textbf{poroelastic}: Enable poroelastic media. This requires a special choice of material properties (see section \ref{subsec:matprop}).
    			\item \textbf{fluidn}: Number of immiscible fluids (1 or 2).
    			\item \textbf{calculate\_tortuosity}: If true, the material parameter tortuosity will be calculated according to $T = 1+r\left(1-\frac{1}{\varphi}\right)$, where $\varphi$ is the porosity and $r$ is a geometrical factor that is 0.5 for spheres \citep[see][for more information]{Boxberg.2019}.
    			\item \textbf{extmatprop}: If true, the material file specified at external\_material\_name will be used, else, matprop will be used.
    			\item \textbf{extmatpropfilename}: Path and filename of the external material property file.
    		\end{itemize}		

		\subsubsection{Adjoint inversion}
			This part of the parfile is displayed in listing \ref{lis:adjpar}. If this is enabeld, an iterative full waveform inversion using the adjoint method is performed. For a detailed description on the method please refer to \cite{Lamert.2020}.
			\lstinputlisting[
    float=ht, 
    captionpos=b, 
    label=lis:adjpar,
    firstnumber = 79,
    firstline = 79,
    lastline = 85,
    caption= Parameters regarding full waveform inversion,
    breaklines=true,
    postbreak=\mbox{\textcolor{red}{$\hookrightarrow$}\space}    
    ]{source_files/parfile.txt}
    		\begin{itemize}
    	 		\item \textbf{inversion}: Enables an iterative full waveform inversion using the adjoint method when set to ``.true.''. This requires an additional parameter file (see section \ref{subsec:invpar} and files with measured data (see section \ref{sec:ininvfiles}).
    	 		\item \textbf{time\_shift}: Overrides the time axis shift of ``shift\_sources'' by a higher value to prepare the source-time-function for low-pass filtering. Can also be used for single forward simulations to prepare synthetic measurements. Please enter a suitable ``lowfreq'' in this case.
    	 		\item \textbf{inv\_steps}: Numer of iteration steps to be performed. If a sufficient misfit reduction is reached before this number of iterations is performed, the program can just be terminated.
    	 		\item \textbf{maskradius}: Defines the radius of the mask around seismic sources, receivers and free surfaces where the values of the search direction are cut out. The entered value is the proportion of the longest dimension of the simulation domain which is used as radius.
    	 		\item \textbf{lowfreq}: The lowest cutoff frequency which is used for low-pass filtering  during the whole inversion. Important to obtain the shift for ``time\_shift''.
    	 		\item \textbf{highfreq}: The maximum cutoff frequency which is used for low-pass filtering  during the whole inversion.
    		\end{itemize}
    				 	
		\subsubsection{Global parameter for the sources}
			This part of the parfile is displayed in listing \ref{lis:srcpar}.			
			\lstinputlisting[
    float=ht, 
    captionpos=b, 
    label=lis:srcpar,
    firstnumber = 87,
    firstline = 87,
    lastline = 88,
    caption= Sources,
    breaklines=true,
    postbreak=\mbox{\textcolor{red}{$\hookrightarrow$}\space}    
    ]{source_files/parfile.txt}
    		\begin{itemize}
    	 		\item \textbf{shift\_sources}: If true, the sources will be shifted by 1.2/f0, so that the maximum is at t = simt0 + delay + 1.2/f0, otherwise the maximum of the wavelet is at t = simt0 + delay (see section \ref{subsec:source}).
    		\end{itemize}
    
		\subsubsection{Global parameter for the stations}
			This part of the parfile is displayed in listing \ref{lis:recpar}.			
			\lstinputlisting[
    float=ht, 
    captionpos=b, 
    label=lis:recpar,
    firstnumber = 90,
    firstline = 90,
    lastline = 92,
    caption= Receiver angle,
    breaklines=true,
    postbreak=\mbox{\textcolor{red}{$\hookrightarrow$}\space}    
    ]{source_files/parfile.txt}
    	\begin{itemize}
    	 \item \textbf{rec\_angle}: This parameter selects the angle by which \textbf{all} stations are rotated. For $0\degree$ the receiver points in the positive z-direction and the angle goes anti-clockwise, i.e., for $90\degree$ the receiver points in negative x-direction.
    	\end{itemize}
    \subsection{source}
    \label{subsec:source}
    	The example displayed in listing \ref{lis:source}, shows a typical source parameter file containing one source. 
    \lstinputlisting[
    float=ht, 
    captionpos=b, 
    label=lis:source,
    caption= Typical source file,
    breaklines=true,
    postbreak=\mbox{\textcolor{red}{$\hookrightarrow$}\space}    
    ]{source_files/source.txt} 
    	\begin{itemize}
    		\item \textbf{nsrc}: This parameter specifies the total number of sources in the model.
    		\item \textbf{source}: Number of the current source. If more than one source is in the file, these number must increase sequentially.
    		\item \textbf{xsource}: X-coordinate of the source. Needs to be inside the boundary of the model, but not inside the PML (if activated).
    		\item \textbf{zsource}: Z-coordinate of the source. Needs to be inside the boundary of the model, but not inside the PML (if activated).
    		\item \textbf{delay}: Parameter to specify a time-delay for the activation of the source.
    		\item \textbf{sourcetype}: Select what type of source is used. Options are: ``0'' for a single force solution and ``1'' for a moment-tensor.
    		\item \textbf{stf}: Select the source time function. Currently the following wavelets are available: ``1'' selects a Gaussian-function, ``2'' selects a Ricker-wavelet, ``3'' a cubed-sine function and ``4'' enables the user to input an arbitrary discrete external wavelet.
    		\item \textbf{extwavelet}: Path and filename of an external wavelet. Is only used if option ``4'' is selected for ``stf''.
    		\item \textbf{f0}: Sets the central frequency of the source wavelet .
    		\item \textbf{factor}: Variable to scale the amplitude of the source time function.
    		\item \textbf{angle\_force}: This parameter is used if a single force solution for the source is used. It sets the direction of the force action. For $0\degree$ the source points in the positive z-direction and the angle goes anti-clockwise, i.e., for $90\degree$ the source points in negative x-direction.
    		\item \textbf{Mxx, Mzz, Mxz}: Components of the moment-tensor, $M=\left(\begin{smallmatrix} M_{xx} & M_{xz} \\ M_{xz} & M_{zz} \end{smallmatrix}\right)$. Used only for a moment-tensor source (sourcetype = 1).
    	\end{itemize}
    	
    	If an arbitrary defined wavelet is chosen by the user, the file containing the information on the wavelet will have to be designed as follows: Two columns separated by spaces. The first column contains the time and the second column contains the amplitude of the waveform.
 
 \medskip
    	When selecting the central frequency ($f_0$) of the source(s) the user has to keep in mind that this frequency directly influences the stability of the simulation, as the maximum length of an element directly depends on $f_0$. Assuming that 10 grid points per wavelength are sufficient to achieve stability, the maximum edge length, $l$, for a given frequency is calculated according to
    	\begin{align*}
    		l = \frac{v_{min}}{\frac{10}{N} * f_{max}}\,,
		\end{align*}
		where $v_{min}$ is the slowest velocity and $N$ is the polynomial order (default is $N = 4$). $N$ can be changed in \url{src/constants.h}. Afterwards the program has to be compiled again. $f_{max}$ is the maximum frequency selected for the source(s).  	   
    \subsection{stations}
    \label{subsec:stations}
    	This example shown in listing \ref{lis:rec}, shows a typical stations parameter file containing ten stations. If not specificly desired by the user, stations should be placed in model space that is not part of the PML (if enabled).
    \lstinputlisting[
    float=ht, 
    captionpos=b, 
    label=lis:rec,
    caption= Typical stations file,
    breaklines=true,
    postbreak=\mbox{\textcolor{red}{$\hookrightarrow$}\space}    
    ]{source_files/stations.txt} 
    
			\begin{itemize}
				\item \textbf{nrec}: This parameter specifies the total number of stations in the model.
				\item \textbf{No xrec zrec}: The first parameter is a running number of the receiver. The other two values represent the x- and z-coordinate of the individual station.
			\end{itemize}			    		 		
    \subsection{fracs}
    \label{subsec:fracs}
    	This is the first parameter file that is added as part of the feature to calculate the influence of fractures on the wave-field. This parameter file specifies the location of one or more fractures and is used if the global parameter ``lsi'' is enabled. A sample file is displayed in listing \ref{lis:fracs}.
    	\lstinputlisting[
    float=ht, 
    captionpos=b, 
    label=lis:fracs,
    caption= Parameter file defining the location of fractures,
    breaklines=true,
    postbreak=\mbox{\textcolor{red}{$\hookrightarrow$}\space}    
    ]{source_files/fracs.txt} \\
    	The first entry after ``BEGIN'' gives the total number of fractures to be read in. \\
    	Each fracture is defined by the following parameters (from left to right)
    	\begin{itemize}
    		\item \textbf{No}: Running number of the fracture.
    		\item \textbf{Property index}: Indicates the type of interface that creates the fracture. The interface is selected from the selection given in ``interfaces''. For example: If there are 4 different slip interfaces given in ``interfaces'' this number may be either 1, 2, 3 or 4. 
    		\item \textbf{startx, startz}: x and z coordinates of the starting point of the fracture.
    		\item \textbf{endx, endz} x and z coordinates of the end point of the fracture.
    	\end{itemize}
    \subsection{interfaces}
    \label{subsec:interfaces}
    	This is the second parameter file that is added as part of the feature to calculate the influence of fractures on the wave-field. This parameter file specifies the properties that can be selected for a fracture via the property index in the previous file. A sample file is displayed in listing \ref{lis:interface}.
    	\lstinputlisting[
    float=ht, 
    captionpos=b, 
    label=lis:interface,
    caption= Parameter file defining fracture properties,
    breaklines=true,
    postbreak=\mbox{\textcolor{red}{$\hookrightarrow$}\space}    
    ]{source_files/interfaces.txt} \\
    	The first entry after ``BEGIN'' gives the total number of properties to be read in. \\
    	Each fracture property is defined by the following parameters (from left to right):
    	\begin{itemize}
    		\item \textbf{type}: currently only ``elastic'', viscoelastic interfaces are under development.
    		\item $\mathbf{\nu_N}$: Relaxationfrequency of the part of the interface normal to the fracture plane.
    		\item $\mathbf{\nu_T}$: Relaxationfrequency of the part of the interface tangential to the fracture plane. \citep[see][for details]{Moeller.2018}.

    	\end{itemize}
   	\subsection{invpar}
    \label{subsec:invpar}
    	This is a parameter file that needs to be added for the full waveform inversion feature. It controls the inversion parameter that can be changed during the inversion. A sample file is displayed in listing \ref{lis:invpar}.
    	\lstinputlisting[
    float=ht, 
    captionpos=b, 
    label=lis:invpar,
    caption= Parameter file defining the inversion parameters,
    breaklines=true,
    postbreak=\mbox{\textcolor{red}{$\hookrightarrow$}\space}    
    ]{source_files/invpar.txt} 
    	\begin{itemize}
    		\item \textbf{upperfreq}: Defines the current cutoff frequency for the low-pass filter
    		\item \textbf{inv\_type}: Indicates the method to obtain the search direction based on the calculated gradients. 1: Steepest descent, 2: Conjugate gradient, 3: L-BFGS. In the first iteration or directly after increasing upperfreq always ``inv\_type'' = 1 is automatically used
    		\item \textbf{step\_1,..,step\_3}: Three test step lengths to obtain the misfit mimimum. For ``inv\_type'' = 1,2 they represent the portion of the maxmimum wave velocity values of the current model. For ``inv\_type'' = 3 they represent a factor for search direction obtained from the L-BFGS method. In the first iteration or directly after increasing upperfreq default values are used.
    		\item \textbf{min\_step, max\_step}: Minimum and maximum test step lengths for ``inv\_type'' = 1,2 that are allowed to prevent infinite loops.
    		\item \textbf{min\_step\_BFGS, max\_step\_BFGS}: Minimal and maximal test step lengths for ``inv\_type'' = 3 that are allowed to prevent infinite loops.
    		\item \textbf{min\_vp, max\_vp, min\_vs, max\_vs}: Defines the minimum and maximum wave velocity values that are allowed in the model after a model change.
    	\end{itemize}
    \section{Files in \texttt{mesh}}
    \label{sec:cubfiles}
    	This section explains the files related to the mesh. These files are designed in a certain way that is expected by NEXD. 
    	In general, any software can be used to create a mesh, but currently there are only scripts available that convert meshs created by Gmsh and Cubit/Trelis to a format that is understood by NEXD.
    	\subsection{coord}
    	\label{subsec:coord}
    	\lstinputlisting[
    float=ht, 
    captionpos=b, 
    label=lis:coord,
    caption=Excerpt from a coord file,
    breaklines=true,
    firstline = 1,
    lastline = 5,
    postbreak=\mbox{\textcolor{red}{$\hookrightarrow$}\space}    
    ]{source_files/coord.txt}
    		This file contains a list that maps the number of a certain node in the mesh to its coordinates. The first entry is the total number of nodes in the mesh. All subsequent lines contain the running number of the nodes, the x-, and the z-coordinate.

		\medskip
    		\fcolorbox{red}{white}{\parbox{\textwidth-2\fboxsep-2\fboxrule}{\textbf{Attention}: Note that the coordinates referred to as z-coordinates are actually the y-coordinates in a Trelis/Cubit model. This is done with regard to the convention that in seismology the depth axis is always the z-axis. If Gmsh is used, please use the xz-plane. }}
    	\subsection{absorb/free}
    	\label{subsec:absfree}
    		\lstinputlisting[
    float=ht, 
    captionpos=b, 
    label=lis:absfree,
    caption=Excerpt from a absorb/free file,
    breaklines=true,
    firstline = 1,
    lastline = 5,
    postbreak=\mbox{\textcolor{red}{$\hookrightarrow$}\space}    
    ]{source_files/absorb.txt} 
    		These two files are set up identically. These files contain the number associated with the nodes that are used to calculate the simple absorbing BC or free surface BC. The first entry in each file is the total number of nodes contained in this file. Additional lines are added sequentially for each node. An example of the first few lines from such a file is shown in listing \ref{lis:absfree}.
    	\subsection{matprop}
    	\label{subsec:matprop}
    		\lstinputlisting[
    float=ht, 
    captionpos=b, 
    label=lis:matprop,
    caption=Entries for a matprop file with (visco-)elastic materials,
    breaklines=true,
    postbreak=\mbox{\textcolor{red}{$\hookrightarrow$}\space}    
    ]{source_files/matpropElastic.txt}
    		The file in listing \ref{lis:matprop} lists the properties of the (visco-)elastic materials used in the model and the file in listing \ref{lis:matpropPoroelastic} lists the properties of the poroelastic materials. First the file scheme for (visco-)elastic simulations is presented:
    		
		\medskip
    		The first entry lists the total number of materials. 
    		The subsequent lines contain the following values for (visco-)elastic materials (from left to right):
    		\begin{itemize}
    			\item Block number (Cubit block or physical material in Gmsh)
    			\item Index identifying an elastic material (1).
    			\item Compressional (P-) wave velocity ($v_\mathrm{P}$).
    			\item Shear (S-) wave velocity ($v_\mathrm{S}$).
    			\item Density ($\rho$).
    			\item Quality factor regarding P-waves ($Q_\mathrm{P}$).
    			\item Quality factor regarding S-waves ($Q_\mathrm{S}$). 
    		\end{itemize}
    		The latter two values are related to viscoelastic calculations and influence attenuation. If attenuation is not selected in the parfile, these values will be irrelevant. If attenuation is selected and a quasi-elastic material is to be part of the simulation, these variables will need to have a high value ($Q \rightarrow \infty$ for an elastic material in reality, $Q=9999$ for the simulation is sufficient).

		\medskip
    		\fcolorbox{red}{white}{\parbox{\textwidth-2\fboxsep-2\fboxrule}{\textbf{Attention}: This file needs to contain ar least two materials if PML are included in the mesh. PML are represented by an individual material within this file. See appendix~\ref{chap:model} for more details}}

			\medskip  
  			In the following, the scheme applying poroelastic materials is shown:
  			
		    \lstinputlisting[
    float=ht, 
    captionpos=b, 
    label=lis:matpropPoroelastic,
    caption=Entries for a matprop file with poroelastic materials,
    breaklines=true,
    postbreak=\mbox{\textcolor{red}{$\hookrightarrow$}\space}    
    ]{source_files/matpropPoroelastic.txt}
		    Also here, the first entry gives the total number of materials. The subsequent lines can be built by nine different models: 
		    
			\begin{itemize}	    
\item Block number, 1, rhos, lambda\^{}u, my, phi, kappa, b,  1/T, 1/N, rho1, S1, K1, ny1, rho2, S2, K2, ny2, fitting\_n, fitting\_chi, Sr1, Sr2
\item Block number, 2, rhos, lambda\^{}u, my, phi, kappa, b,  1/T, 1/N, rho1, S1, K1, ny1, rho2, S2, K2, ny2, p\_b,       lambda\_BC,   Sr1, Sr2
\item Block number, 3, rhos, K\^{}d,      my, phi, kappa, 0., 1/T, Ks,  rho1, S1, K1, ny1, rho2, S2, K2, ny2, fitting\_n, fitting\_chi, Sr1, Sr2
\item Block number, 4, rhos, K\^{}d,      my, phi, kappa, 0., 1/T, Ks,  rho1, S1, K1, ny1, rho2, S2, K2, ny2, p\_b,       lambda\_BC,   Sr1, Sr2
\item Block number, 5, rhos, K\^{}d,      my, phi, kappa, 0., 1/T, 1/N, rho1, S1, K1, ny1, rho2, S2, K2, ny2, fitting\_n, fitting\_chi, Sr1, Sr2
\item Block number, 6, rhos, K\^{}d,      my, phi, kappa, 0., 1/T, 1/N, rho1, S1, K1, ny1, rho2, S2, K2, ny2, p\_b,       lambda\_BC,   Sr1, Sr2
\item Block number, 7, rhos, lambda\^{}u, my, phi, kappa, b,  1/T, 1/N, rho1, S1, K1, ny1, rho2, S2, K2, ny2, A,         0.,          Sr1, Sr2
\item Block number, 8, rhos, K\^{}d,      my, phi, kappa, 0., 1/T, Ks,  rho1, S1, K1, ny1, rho2, S2, K2, ny2, A,         0.,          Sr1, Sr2
\item Block number, 9, rhos, K\^{}d,      my, phi, kappa, 0., 1/T, 1/N, rho1, S1, K1, ny1, rho2, S2, K2, ny2, A,         0.,          Sr1, Sr2
\end{itemize} 		    
		    The second entry of each line defines the used model. The different parameters are: 
		    \begin{itemize}
		    \item Block number (Cubit block or physical material in Gmsh)
		    \item rhos = density solid
		    \item lambda\^{}u = first Lamé parameter (undrained)
		    \item K\^{}d = bulk modulus of skeleton (drained)
		    \item mu = shear modulus
		    \item phi = porosity
		    \item kappa = permeability
		    \item b = biot coefficient
		    \item 1/T = inverse tortuosity (Note, that this is replaced by r, if 1/T is calculated! See \ref{subsec:parfile})
		    \item 1/N = inverse of biot modulus
		    \item Ks = bulk modulus of solid grain material
		    \item rho1 = density fluid 1
		    \item S1 = saturation fluid 1
		    \item K1 = bulk modulus fluid 1
		    \item ny1 = viscosity fluid 1 (this is the wetting fluid)
		    \item rho2 = density fluid 2
		    \item S2 = saturation fluid 2
		    \item K2 = bulk modulus fluid 2
		    \item ny2 = viscosity fluid 2 (this is the non-wetting fluid)
		    \item fitting\_n, fitting\_chi = fitting parameters for van Genuchten model n, chi \citep[see][]{Boxberg.2019}
		    \item p\_b = bubbling pressure
		    \item lambda\_BC = fitting parameter for Brooks \& Corey model, see \citep{Boxberg.2019})
		    \item A = capillary pressure coefficient for Douglas Jr. et al. model, see \citep{Boxberg.2019})
		    \item Sr1 = residual saturation fluid 1
		    \item Sr2 = residual saturation fluid 2
		    \end{itemize} 
		    		  
		  \lstinputlisting[
    float=htp, 
    captionpos=b, 
    label=lis:matpropporo,
    caption=Structure of a matpropporo file,
    breaklines=true,
    postbreak=\mbox{\textcolor{red}{$\hookrightarrow$}\space}    
    ]{source_files/matpropporo.txt}
    		  Instead of creating this material file, a matpropporo file can be specified (necessary for the Matlab code), which looks slightly different (see \ref{lis:matpropporo}).
  
    	\subsection{mat}
    	\label{subsec:mat}
    		\lstinputlisting[
    float=ht, 
    captionpos=b, 
    label=lis:mat,
    caption=Excerpt from the mat file,
    breaklines=true,
    firstline = 1,
    lastline = 5,
    postbreak=\mbox{\textcolor{red}{$\hookrightarrow$}\space}    
    ]{source_files/mat.txt}
    		This file maps the material properties from the matprop file to the elements. Each line consists of the running element number, the ID of the block (in the ``matprop'' file) and an index to tell if the element is part of a PML. The latter index is either ``0'' if it is part of the regular medium or ``1'', if it is part of a PML. 
    	\subsection{mesh}
    	\label{subsec:mesh}
    		\lstinputlisting[
    float=ht, 
    captionpos=b, 
    label=lis:mesh,
    caption=Excerpt from the mesh file,
    breaklines=true,
    firstline = 1,
    lastline = 5,
    postbreak=\mbox{\textcolor{red}{$\hookrightarrow$}\space}    
    ]{source_files/mesh.txt}
    		This file contains a mapping that relates the nodes and the material properties to the individual element. The first entry is the total number of elements in the mesh. Afterwards, each line contains the following entries (from left to right):
    		\begin{itemize}
    			\item Running number of the element
    			\item The numbers of the three nodes that are used to construct the element.
	   		\end{itemize}
	   		A few lines from a mesh file are shown in code listing \ref{lis:mesh}.
    \section{Files in \texttt{inversion}}
    \label{sec:ininvfiles}
    	Files in \texttt{inversion} are only necessary when performing a full waveform inversion. The measured data which shall be reconstructed during the inversion are saved here as displacement seismograms. For the current version of the code the measured data need to be complete. That means for every source a measured seismogram must exist for every station and every component.
    	
	\medskip    
    	The filename of the measurements is similar to the output files of the solver (see section \ref{sec:solver}):
    	
    	\begin{center}%[4cm]{4cm}% 1em left, 2em right
				\emph{seismo.(component).(number of the station).sdu.(number of the source).meas}
		\end{center}
		
		The component is either ``x'' or ``z''. The number of the station is a seven-digit number (e.g. 0000001) associated to the number of the station in the \texttt{station} file. The number of the source is a three-digit number (e.g. 001) corresponding to the number of the source in the \texttt{source} file. 
		
		Please make sure that the time axis of the measurements completely covers the time axis of the simulation.
		
		To obtain synthetic measurements from a forward simulation, a Python script \texttt{copy\_seis.py} is provided in the folder \texttt{tools}. It copies the seismograms from the folder \texttt{out} into the folder \texttt{inversion} and adjusts the file name by the source number which needs to be entered in the Python script.
    	